\chapter{Humanoide Roboter}\label{sec:humanoide-roboter}
\section{Allgemein}\label{sec:allgemein}
Erklärung was humanoide Roboter sind.

\section{Abgrenzung zu anderen Arten von Robotern}\label{sec:abgrenzung}
Häufig ist der Unterschied zwischen humanoiden Robotern und anderen Arten von
Robotern nicht ganz klar. So verschwimmen teilweise die Grenzen zwischen
Industrierobotern, Servicerobotern, humanoiden Robotern und Androiden.

\subsection{Industrieroboter}\label{sec:industrieroboter}
Der Unterschied zwischen humanoiden Robotern und Industrierobotern ist relativ
deutlich. Industrieroboter werden dazu entwickelt einzelne Schritte oder den
gesamten Fertigungsprozess zu automatisieren. Ihre Bewegungsabläufe erinnern oft
an die eines Menschen, da Industrieroboter meist aus mehreren Gelenken bestehen,
die unabhängig voneinander bewegt werden können, ähnlich wie der menschliche
Arm. \cite{Weber2017} Bei Industrierobotern wird allerdings kein besonderer
Fokus auf menschliche Bewegungen gelegt. Im Gegensatz zu humanoiden Robotern
werden Industrieroboter nicht entwickelt um einem Menschen ähnlich zu sein,
sondern um möglichst effizient arbeiten zu können.

\subsection{Serviceroboter}\label{sec:serviceroboter}
Unter Servicerobotern versteht man Roboter, die eine Aufgabe größten Teils
autonom erledigen. Bei diesen Aufgaben handelt es sich um Dinge, die ein Mensch
nicht erledigen kann oder will. Weit verbreitete Serviceroboter sind z.~B.
Staubsaugroboter oder Mähroboter. Der Unterschied zu humanoiden Robotern besteht
hierbei darin, dass Serviceroboter nur eine Art von Aufgaben erledigen können.
Außerdem ist ihr Aussehen an die zu erledigende Aufgabe angepasst, nicht an das
Aussehen eines Menschen.

\subsection{Androiden}\label{sec:androiden}
Bei Androiden handelt es sich um humanoide Roboter, deren Zweck es ist, Menschen
möglichst gut nachzuahmen. Wenn über das Thema Roboter im Allgemeinen gesprochen
wird, denken viele Menschen zuerst an Androiden oder humanoide Roboter auch wenn heute
Industrieroboter viel weiter verbreitet sind und die Chance z.~B. einen
Mähroboter zu besitzen deutlich höher ist, als einen Androiden zu Hause zu
haben. Dass Androiden trotzdem so verbreitet in der Vorstellung der Menschen
sind, liegt zum Einen an Büchern, Serien und Filmen und zum Anderen an der
Faszination, die Androiden auslösen. \cite{Dautenhahn2011}

\subsubsection{Androiden in Medien}
In vielen Science-Fiction Büchern, Serien und Filmen, sind Androiden oder
humanoide Roboter ein wesentlicher Bestandteil der Geschichte. Dabei treten sie
in sehr unterschiedlichen Arten auf. Jedoch haben alle gemeinsam, dass sie in
nicht all zu ferner Zukunft existieren und einen gewissen Einfluss auf das Leben
der Menschen haben. Dieser Einfluss ist in manchen Darstellungen positiv, in
anderen negativ.

\subparagraph{}
Geht es nach den Vorstellungen der Autoren,
scheint ein Leben ohne humanoide Roboter in Zukunft nicht vorstellbar zu sein.
Im realen Leben ist die Wahrscheinlichkeit mit einem Industrieroboter oder einem
Serviceroboter in Kontakt zu kommen zwar deutlich höher als auf einen Androiden
zu treffen, durch das häufige Auftreten in den Medien sind Androiden jedoch
sehr weit in der Vorstellung der Menschen vertreten.

\subsubsection{Faszination von Androiden}
Obwohl Serviceroboter oder Industrieroboter technisch sehr aufwändig sein
können, haben sie für einen einzelnen Menschen meist keine sehr große Bedeutung.
Ein Mähroboter ist zwar nützlich, jedoch tut er von außen betrachtet nichts
anderes als den ganzen Tag über den Rasen zu fahren. Mit Industrierobotern
kommen Menschen noch seltener in Kontakt, wenn sie nicht gerade in einer Firma
arbeiten, in der diese eingesetzt werden. Auch diese Industrieroboter erledigen
meist nur eine einzelne Aufgabe und sind für Außenstehende auf Dauer wenig
interessant.

\subparagraph{}
Androiden lösen hier eine viele größere Faszination aus. Sie
stellen eine Zukunftsvision dar, die in der Realität noch kaum vertreten ist.
Viele Menschen sind neugierig, was Androiden alles können und sind fasziniert
von ihren Fähigkeiten. Außerdem können Menschen mit Androiden interagieren.
Androiden können auf Fragen antworten, auf Berührungen reagieren und die
Menschen unterhalten. Dadurch sind Androiden auf Dauer viel spannender als nicht
interaktive Industrie- oder Serviceroboter.

\subsubsection{Androiden in der Forschung}
Trotz der weiten Verbreitung in Medien und der Vorstellung von
Menschen, machen Androiden nur einen kleinen Teil der Forschung innerhalb der
viel größeren Bereiche Robotik und Künstliche Intelligenz aus.
\cite{Dautenhahn2011} Trotzdem arbeiten verschiedene Unternehmen an der
Entwicklung von humanoiden Robotern und Androiden. Beipiele hierfür sind Pepper
(Kapitel \ref{sec:pepper}) und Nao von SoftbankRobotics und Aldebaran oder Asimo
von Honda, welcher als der am weitesten entwickelte humanoide Roboter der Welt
beworben wird. \cite{Honda2018}

\section{Moderne Humanoide Roboter}
Heute gibt es verschiedene humanoide Roboter, die grundsätzlich für die
Vorstellung von PowerPoint-Folien einsetzbar sind.

\subsection{Atlas}
Atlas wird von der Firma Boston Dynamics produziert. Boston Dynamics ist bereits
bekannt für seine laufenden Roboter, die sich auch in unwegsamen Gebieten sicher
fortbewegen können. Während andere Roboter von Boston Dynamics Tieren
nachempfunden sind, ist Atlas der erste Roboter der sich wie ein Mensch bewegt
und aussieht. 

\subsection{Sophia}
Das Unternehmen Hanson Robotics arbeit daran den menschlichsten Roboter, der zur
Zeit existiert, zu entwickeln. Sophias Silikonhaut ist kaum von menschlicher
Haut zu unterscheiden. Der Roboter kann über 62 Gesichtsausdrücke darstellen.
Mit Kameras in den Augen, kann Sophia Menschen verfolgen und Augenkontakt zu
ihnen aufnehmen. Zusammen mit verschiedenen Technologien von Google, IBM und
Intel kann Sophia Sprache erkennen und sie verarbeiten. Außerdem lernt sie
dazu. \cite{Harriet2016} Auf der "`Future Investment Initiative"', einer
Konferenz in Saudi-Arabien wurde Sophia der Öffentlichkeit präsentiert. Dabei
wurde sie von einem Moderator interviewt. Das Gespräch mit Sophia wirkt
natürlich. Während dieser Konferenz wurde Sophia die arabische
Staatsbürgerschaft verliehen, dies ist allerdings eher als PR-Gag zu verstehen,
als dass Sophia soweit entwickelt ist, dass sie die Staatsbürgerschaft wirklich
verdient. \cite{Welt2017}

\subsection{Pepper}\label{sec:pepper}
Pepper ist ein humanoider Roboter der Firmen SoftBank Robotics und Aldebaran.
Vom Hersteller wird er als freundlich, liebenswert und überraschend beworben.
Entwickelt wurde Pepper um ein alltäglicher Begleiter für Menschen zu sein. Er
wird als "`viel mehr als ein Roboter"' beschrieben. Mit seiner Körpersprache und
der Stimme soll er auf die "`natürlichste und intuitivste"' Art mit Menschen
kommunizieren. \cite{SoftBank2018}

\subparagraph{}
Mit einer Größe von etwa 1,20m, seinem kindlichen Gesicht und beweglichen Armen
und Händen, soll Pepper Menschen glücklich machen. Sein Kopf bewegt sich, um
vorbeilaufende Passanten anzuschauen oder einem Menschen in die Augen zu
schauen, wenn er mit dem Roboter spricht. \cite{Markowitz2015}

\subparagraph{}
Das Design von Pepper ist darauf ausgelegt Emotionen sowohl zu verstehen als
auch auszulösen. Die dazu verwendeten Eigenschaften sind auch wichtig für das
Vorstellen von PowerPoint Präsentation.

\subsubsection{Hören und Sprechen}\label{sec:hoeren-und-sehen}
Mit Lautsprechern und Mikrofonen ausgestattet, kann Pepper gesprochenes
Verstehen und darauf reagieren. In der Anwendung soll Pepper die Notizen einer
PowerPoint Präsentation vorlesen, wozu die Mikrofone benutzt werden. Die
Fähigkeit zuzuhören soll dazu verwendet werden Pepper während der Präsentation
Anweisungen geben können. Er soll auf Befehle wie "`Pause"' und "`Weiter"'
reagieren und die Präsentation entsprechen pausieren bzw. fortsetzen.

\subsubsection{Sehen}\label{sec:sehen}
Eine 3D-Kamera und zwei HD-Kameras helfen Pepper seine Umgebung zu analysieren
und Menschen und Bewegungen zu erkennen. Während einer Präsentation wird diese
Fähigkeit, außer bei der schon vom Hersteller implementierten
Kollisionserkennung, nicht verwendet. Es wird nicht nötig sein spezielle Dinge
zu erkennen.

\subsubsection{Verbindung}\label{sec:verbindung}
Pepper verfügt über eine Internetverbindung. Dies ist wichtig, da die PowerPoint
Präsentation vor der Vorstellung durch Pepper nicht zuerst auf den Roboter
geladen werden soll. Die Informationen wie die einzelnen Folien und der
zugehörige Text sollen auf einem Server bereitgestellt werden auf den der
Roboter zugreifen kann. Mehr hierzu in Kapitel \ref{sec:umsetzung} auf Seite
\pageref{sec:umsetzung}.

\subsubsection{Tablet}\label{sec:tablet}
Auf der Brust von Pepper ist ein Tablet angebracht, auf dem Bilder, Videos oder
Webseiten angezeigt werden können. Auf diesem Tablet sollen die einzelnen Folien
der PowerPoint Präsentation angezeigt werden. \cite{SoftBankII2018}

\section{Human-Robot Interaction}\label{sec:hri}
\ac{hri} ist ein relativ neues, aber wachsendes Forschungsgebiet, welches sich
mit der Interaktion zwischen Menschen und Robotern befasst. Außerdem soll
herausgefunden werden wie Roboter am besten mit Menschen zusammenarbeiten
können. Dabei hat \ac{hri} nicht nur Einfluss auf die Wirtschaft, sondern auch
auf mögliche Arten der Beziehungen zwischen Menschen und Robotern. Deshalb
verbindet \ac{hri} verschiedene Wissenschaften, wie Psychologie und
Sozialwissenschaften mit Informatik und Robotik. Eines der Hauptziele von
\ac{hri} ist das Erforschen möglichst natürlicher Wege der Kommunikation
zwischen Menschen und Robotern. \cite{Dautenhahn2011}

\subsection{Natürliche Formen der
Kommunikation}\label{sec:nat-formen-kommunikation}
Um eine natürliche Kommunikation zwischen Menschen und Robotern zu ermöglichen,
müssen mehrere, von einem Benutzer potenziell verwendbare, Interfaces
bereitgestellt werden. Dabei werden klassische mit neueren Interfaces
kombiniert.

\subsubsection{Graphisches Interface}
Ein klassisches Interface zur Interaktion mit Maschinen sind graphische
Input-Output Schnittstellen. Wie bereits von der Kommunikation mit Computern
bekannt, könnte ein Roboter durch Eingaben mit Maus und Tastatur gesteuert
werden. Diese Form des Interfaces ist durch die weite Verbreitung von Computern
den meisten Menschen bekannt. Allerdings sind Maus und Tastatur keine
natürliche Art der Kommunikation. Dieses Interface kann durch die Verwendung
eines Touchscreens intuitiver und damit auch natürlicher gestaltet werden.

\subparagraph{}
Vorteil von graphischen Interfaces ist, dass der Benutzer dem Roboter klare
Instruktionen geben kann. über Menüs muss der Benutzer eine bestimmte Aktion
auswählen, die der Roboter ausführen soll. Der Roboter führt dann den
entsprechenden Programmcode aus. Auf diese Weise gibt es keinen Spielraum für
eventuelle Fehlinterpretationen der Instruktionen durch den Roboter.

\subsubsection{Sprache}
Sprache ist eine sehr natürliche Form der Kommunikation. Ein Mensch kann einem
Roboter einen Befehl geben, indem er dem Roboter sagt, welche Aktion er
ausführen soll. Der Mensch ist es gewohnt durch das Sprechen zu kommunizieren,
deshalb ist Sprache sowohl ein natürliches als auch ein intuitives Interface.

\subparagraph{}
Allerdings ist die Umsetzung dieses Interface komplizierter als die eines
einfachen graphischen Interface. Zunächst muss der Roboter die Worte, die der
Mensch spricht erkennen. Spracherkennung ist bereits weit verbreitet und wird in
verschiedenen Anwendungen verwendet. So lassen sich z.~B. Smartphones mit Siri
oder dem Google Assistant per Sprachbefehlen bedienen. Allerdings muss der
Roboter auch das verstehen, was der Mensch meint. Schon eine vermeintlich
einfache Anweisung, wie z.~B. "`Setz dich."' kann vom Roboter unterschiedlich
interpretiert werden. Er kann sich z.~B. entweder auf einen Stuhl oder auf den
Boden setzen.

\subparagraph{}
Zwar sind Sprachbefehle natürlicher und intuitiver als graphische Eingaben auf
einem Display, dem Menschen, der dem Roboter Anweisungen gibt, wird es jedoch
schwerer fallen seine Anweisungen so präzise zu formulieren, dass der Roboter
genau das tut, was er tun soll. Durch die Auswahl aus einem Menü wäre dies
einfacher.

\subparagraph{}
Außerdem muss der Roboter sich an die Dynamik, die sich aus einer Interaktion
ergibt anpassen. Kommunikation besteht meist nicht nur aus einem Befehl des
Menschen und der darauf folgenden Aktion des Roboters. Der Roboter kann
nachfragen, wenn ein Befehl nicht präzise genug formuliert wird und entsprechend
auf die Antwort des Menschen reagieren. Diese Frage-Antwort Dynamik muss der
Roboter verstehen und Antworten in den richtigen Kontext setzen.

\subsubsection{Visuelle Kommunikation}
Mit einer Kamera kann der Roboter Bewegungen, Gesten und Mimik eines Menschen
erkennen. So kann der Roboter mit der Hand gesteuert werden, reagieren, wenn ein
Mensch Blickkontakt zu ihm aufnimmt. So ist es auch möglich einem Roboter einen
Bewegungsablauf vorzumachen, den er dann wiederholen kann, was das Erklären
einer Aktion deutlich vereinfachen kann.

\subsubsection{Tastsensoren}
In der direkten Interaktion wird es auch zu
Berührungen zwischen Meschen und Robotern oder Robotern untereinander kommen. Um
dabei keine Schäden zu verursachen oder Menschen zu verletzen muss der Roboter
auf Berührungen reagieren können. Berührungen sind eine direkte Form der
Kommunikation, die dem Roboter unmissverständlich bestimmte Anweisungen geben
kann. So kann fest einprogrammiert werden dass sich der Roboter nicht mehr
bewegt, nachdem er berührt wurde. Damit wird verhindert, dass er Menschen
verletzt. \cite{Prassler2004}

\section{Einsatz von humanoiden Robotern}\label{sec:einsatz}
\subsection{Öffentliche Plätze}\label{sec:oeffentliche-plaetze}
Im Incheon Internation Airport in Südkorea werden mehrere humanoide Roboter von
LG eingesetzt um für mehr Sauberkeit und eine leichtere Orientierung der
Passagiere zu sorgen. Dazu werden zwei verschiedene Typen von Robotern
verwendet. Zum einen ein Staubsaugroboter, der mit künstlicher Intelligenz
ausgestattet ist. Er merkt sich welche Bereiche am häufigsten gereinigt werden
müssen und berechnet so ideale Putzrouten.

\subparagraph{}
Zum anderen wird eine größere Variante des, auch für Privathaushalte
verfügbaren, "`Hub Robot"' verwendet. Dieser kann mit Passagieren kommunizieren
und über ein großes Display Informationen anzeigen. Bei Bedarf kann er Fluggäste
persönlich zu einem von ihnen gewählten Zielpunkt begleiten.
\cite{Beineke2017}

\subparagraph{}
Als erster deutscher Flughafen testet der Flughafen München zusammen mit
Lufthansa den Einsatz eines humanoiden Roboters. Dazu wird Pepper, vom Flughafen
"`Josie Pepper"' getauft, verwendet. In einer Testphase wird der Roboter im
Terminal eingesetzt. Es soll herausgefunden werden, wie die Reaktion der
Passagiere auf den Roboter ausfallen. Mit hilfe von IBM Watson ist Pepper an
Daten des Flughafens angebunden und kann so Fragen der Passagiere beantworten.
So kann der Roboter zum Beispiel den Weg zum Abfluggate eines Fluges erklären.
\cite{MunichAirport2018}

\subparagraph{}
Ähnlich wie am Flughafen werden humanoide Roboter auch an Bahnhöfen eingesetzt.
In Frankreich wird Pepper an drei verschiedenen Bahnhöfen verwendet um Reisende
während Wartezeiten zu unterhalten. Außerdem liefert Pepper Informationen zu
Zügen und erfasst die Kundenzufriedenheit. \cite{SoftBankIV2018}

\subsection{Einzelhandel}
Da humanoide Roboter Aufmerksamkeit auf sich ziehen, werden sie gerne verwendet
um Kunden in Läden zu locken, sie auf Produkte hinzuweisen oder dafür zu sorgen,
dass Kunden sich länger im Laden aufhalten. Zum Beispiel wird Pepper in
Fillialen des Herstellers von Pepper, SoftBank, eingesetzt. Außerdem berät
Pepper Kunden in Fillialen von Nestlé und Carrefour. \cite{SoftBankIV2018}
