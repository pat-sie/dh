\chapter{Grundlagen}\label{sec:grundlagen}
\section{Python}\label{sec:python}
Anwendungen für den Roboter Pepper (Kapitel \ref{sec:pepper}) können in Python,
C++, Java, JavaScript oder ROS programmiert werden. \cite{SoftBankIII2018} Bei
den bisherigen Anwendungen wurde Python als Programmiersprache ausgewählt.
Deshalb wird auch für diese Anwendung, neben JavaScript für die Webseite, als
hauptsächliche Programmiersprache verwendet.

\subparagraph{}
 "`Python ist eine portable,
interpretative, objektorientiere Programmiersprache"' \cite{Weigend2017}. Ihre
Entwicklung wurde 1989 von Guido van Rossum begonnen. Heute wird die
Entwicklung von der nichtkommerziellen Organisation "`Python Software Foundation"'
\footnote{\url{https://www.python.org/psf/}} koordiniert.

\subparagraph{}
Das Python-Skript wird von einem Interpreter ausgeführt. Python-Skripte können
auf verschiedenen Systemplattformen (Unix, Windows, Mac OS) laufen, deshalb
bezeichnet man Python als portable Sprache. Zudem kann mit Python sehr gut
lesbarer Code geschrieben werden und ein Programm kann mit weniger Code erstellt
werden als in anderen Programmiersprachen. Listing \ref{lst:python-hello-world}
zeigt ein Hello World Programm in Python im Vergleich zu einem Java-Programm in
Listing \ref{lst:java-hello-world}, welches die gleiche Funktion hat.
Die aktuelle Version ist 3.6.
Da auf dem Roboter allerdings Python 2.7 installiert ist, wird die Anwendung mit dieser
Version entwickelt. \cite{Weigend2017}

\begin{lstlisting}[float, language=Python, frame=single, framexleftmargin=15pt,
style=algoBericht, label={lst:python-hello-world}, captionpos=b, caption={Hello
World in Python}] 
print "Hello, world!"
\end{lstlisting}

\begin{lstlisting}[float, language=Java, frame=single, framexleftmargin=15pt,
style=algoBericht, label={lst:java-hello-world}, captionpos=b, caption={Hello
World in Java}] 
public class HelloWorld
{
	public static void main(String[] args)
	{
		System.out.println("Hello, world!");
	}
}
\end{lstlisting}

\section{JavaScript}\label{sec:javascript}
%TODO
JavaScript ist, wie der Name bereits sagt, eine Skriptsprache. Verwendet wird
sie hauptsächlich im World Wide Web. Mit JavaScript lassen sich Objekte, wie zum
Beispiel das Browserfenster, beeinflussen. \cite{Steyer2010}

\section{PowerPoint}\label{sec:powerpoint}
Microsoft PowerPoint ist ein sehr weit verbreitetes Programm zum erstellen von
Präsentationsfolien.

%% Ich würde hier etwas über die allgemeine Verbreitung von Powerpoint schreiben
%% und darauf hinweisen, dass auch wir (Fiducia GAD) im Rahmen der Distribution
%% des Bankarbeitsplatzes Powerpoint ausrollen. Es ist daher naheliegend
%% anzunehmen, dass Powerpint ein einfach zu bedienendes Frontend für den
%% Kundenkreis darstellt.
%%
%% http://www.thielsch.org/download/paper/Thielsch_Perabo_2012.pdf

\section{Flask}\label{sec:flask}
Flask ist ein Python Package zum erstellen eines Servers.

\section{Verteilte Systeme}\label{sec:verteilte-systeme}

\subsection{Client-Server-Modell}\label{sec:client-server-modell}
Client-Server-Systeme bestehen aus zwei logischen Einheiten. Diese sind ein,
oder mehrere Clients und ein Server. Die Clients fordern Dienste oder Daten des
Servers an. Der Server stellt die entsprechenden Dienste oder Daten zur
Verfügung.

\subparagraph{}
%TODO
Zusammen bilden diese Einheiten ein komplettes System mit unterschiedlichen
Zuständigkeitsbereichen, wobei diese Zuständigkeiten oder Rollen fest zugeordnet
sind. Entweder ist ein Prozess ein Client oder ein Server. Ein Server kann
mehrer Kunden (Clients) bedienen. Die Kunden eines Servers haben keinerlei
Kenntnis voneinander und stehen demgemäß auch in keinem Bezug zueinander, außer
der Tatsache, dass sie den gleichen Serber verwenden. Clients und Server können
auf dem gleichen oder auf unterschiedlichen Rechnern laufen.

\subparagraph{} 
%TODO
Client und Server sind zwei Ausführungspfade oder -einheiten mit einer
Komsumenten-Produzenten-Beziehung. Clients dienen als Konsumenten und tätigen
Anfragen an Server über Dienste oder Information. Sie benutzen dann die
Rückantwort zu ihrem eigenen Zweck und zur Erledigung ihrer Aufgabe. Server
spielen die Rolle des Produzenten und erledigen die Daten- oder Dienstanfragen,
die von den Clients gestellt wurden. Die Interaktion zwischen den Clients und
dem Server verlaufen somit nach einem fest borgegebenen Protokoll: Der Client
sendet eine Anforderung (Request) an den Server, dieser erledigt die Anforderung
oder Anfrage und schickt eine Rückantwort (Reply) zurück an den Client.

\subparagraph{}
%TODO
Ein Client ist ein auslösender Prozess, und ein Server ist ein reagierender
Prozess. Clients tätigen eine Anforderung, die eine Reaktion des Servers
ausläst. Clients initiieren Aktivitäten zu beliebigen Zeitpunkten, und
andererseits warten Server auf Anfragen von Clients und reagieren dann darauf.
Der Server stellt somit einen zentralen Punkt dar, an den Anforderungen
geschickt werden können, und nach Erledigung der Anfrage sendet der Server das
Ergebnis an den Client zurück.
\cite{Bengel2015}

\section{MQTT}\label{sec:mqtt}
\ac{mqtt} ist ein Protokoll zum Austausch von Nachrichten. Es ist simpel
aufgebaut und wurde für die Verwendung zwischen Geräten mit geringer
Funktionalität entwickelt. Da \ac{mqtt} ein sehr robustes Protokoll ist, eignet
es sich gut für unzuverlässige Netze mit geringer Bandbreite und hoher
Latenzzeit. \ac{mqtt} bietet eine hohe Zuverlässigkeit und minimiert die
genutzte Bandbreite. Durch die Eigenschaften von \ac{mqtt} wird es z.~B. bei
Sensornetzwerken, Machine to Machine, Telemedizin, Patientenüberwachung und
"`Internet der Dinge"' eingesetzt.

\subparagraph{}
\ac{mqtt} ist offen und lizenzfrei. Das Protokoll wurde 1999 von IBM für die
Satellitenkommunikation entwickelt. Später wurde es auch in vielen weiteren
industriellen Anwendungen eingesetzt.

\subparagraph{}
\ac{mqtt} ist einfach zu implementieren. Es kann ununterbrochene Sitzungen mit
Geräten betreiben. Seine Struktur besteht aus einem Publisher, einem Subscriber
und einem Message-Broker, der für die Kommunikationssteuerung sorgt.
\cite{dennisseidel2018}
