\chapter{Grundlagen}\label{sec:grundlagen}
\section{Python}\label{sec:python}
Anwendungen für den Roboter Pepper (Kapitel \ref{sec:pepper}) können in Python,
C++, Java, JavaScript oder ROS programmiert werden. \cite{SoftBankIII2018} Bei
den bisherigen Anwendungen wurde Python als Programmiersprache verwendet.
Deshalb wird auch für diese Anwendung als
hauptsächliche Programmiersprache verwendet. "`Python ist eine portable,
interpretative, objektorientiere Programmiersprache"' \cite{Weigend2017}. Ihre
Entwicklung wurde 1989 von Guido van Rossum begonnen. Heute wird die Entwicklung
von der nichtkommerziellen Organisation "`Python Software Foundation"'
\footnote{\url{https://www.python.org/psf/}} koordiniert.

\subparagraph{}
Das Python-Skript wird von einem Interpreter ausgeführt. Python-Skripte können
auf verschiedenen Systemplattformen (Unix, Windows, Mac OS) ausgeführt werden.
Deshalb bezeichnet man Python als portable Sprache. Zudem gilt Python-Code als
sehr gut lesbar und ein Programm kann mit weniger Code erstellt
werden als in anderen Programmiersprachen. Listing \ref{lst:python-hello-world}
zeigt ein "`Hello World"'-Programm in Python im Vergleich zu einem Java-Programm
in Listing \ref{lst:java-hello-world}, welches die gleiche Funktion hat.
Die aktuelle Python Version ist 3.6.
Da auf dem Roboter allerdings Python 2.7 installiert ist, wird die Anwendung mit
dieser Version entwickelt. \cite{Weigend2017}

\begin{lstlisting}[float, language=Python, frame=single, framexleftmargin=15pt,
style=algoBericht, label={lst:python-hello-world}, captionpos=b, caption={Hello
World in Python}]
print "Hello, world!"
\end{lstlisting}

\begin{lstlisting}[float, language=Java, frame=single, framexleftmargin=15pt,
style=algoBericht, label={lst:java-hello-world}, captionpos=b, caption={Hello
World in Java}]
public class HelloWorld { 
	public static void main(String[] args) {
		System.out.println("Hello, world!"); 
	}
}
\end{lstlisting}

\section{PowerPoint}\label{sec:powerpoint}
Microsoft PowerPoint ist ein sehr weit verbreitetes Programm zum erstellen von
Präsentationsfolien.

\section{Flask}\label{sec:flask}
Flask ist ein Python Package zum erstellen eines Servers.

\section{Verteilte Systeme}

\subsection{Client}
Ein Client ist eine Ausführungseinheit, die mit einem Server in einer
Konsumenten-Produzenten-Beziehung steht. Der Client dient als Konsument. Er
stellt Anfragen zu Diensten oder Informationen an den Server. Die Antwort auf
diese Anfragen werden vom Client zu seinem eigenen Zweck und zur Erledigung
seiner Aufgabe verwendet. Clients sind auslösende Prozesse, auf die ein Server
reagieren kann. Sie können Aktivitäten zu beliebigen Zeiten initiieren.
\cite{Bengel2015}

\subsection{Server}
Der Server dient in der Konsumenten-Produzenten-Beziehung als Produzent. Er
bearbeitet die Daten- und Dienstanfragen, die durch einen Client gestellt
werden. Ein Server ist ein reagierender Prozess. Er wartet auf Anfragen durch
einen Client und reagiert darauf, sobald eine Anfrage gestellt wird. Nach dem
Bearbeiten sendet der Server das Ergebnis an den entsprechenden Client zurück.

Die Interaktion zwischen den Clients und
dem Server verlaufen somit nach einem fest vorgegebenen Protokoll: Der Client
sendet eine Anforderung (Request) an den Server, dieser erledigt die Anforderung
oder Anfrage und schickt eine Rückantwort (Reply) zurück an den Client.

Der Server stellt somit einen zentralen Punkt dar, an den Anforderungen
geschickt werden können
\cite{Bengel2015}

\subsection{Client-Server-Modell}
Client-Server-Systeme bestehen aus einem Server und einem oder mehreren Clients.
Dabei können Clients und Server auf unterschiedlichen aber auch auf dem gleichen
Rechner laufen. Client und Server haben unterschiedliche Zuständigkeitsbereich,
diese sind ihnen jeweils fest zugeordnet. Zusammen bilden diese Einheiten ein
komplettes System. Ein Server kann mehrere Clients bedienen. Die Clients haben
keine Kenntnis voneinander. Der einzige Bezug zueinander ist, dass sie den
gleichen Server verwenden.

\subparagraph{}
Die Interaktion zwischen Client und Server laufen nach einem fest vorgegebenen
Protokoll ab. "`Der Client sendet eine Anforderung (Request) an den Server,
dieser erledigt die Anforderung oder Anfrage und schickt eine Rückantwort
(Reply) zurück an den Client."'
\cite{Bengel2015}

\section{MQTT}\label{sec:mqtt}
\ac{mqtt} ist ein lizenzfreies Protokoll zum Austausch von Nachrichten. Es ist
simpel aufgebaut und wurde für die Verwendung zwischen Geräten mit geringer
Funktionalität entwickelt. Da \ac{mqtt} ein sehr robustes Protokoll ist, eignet
es sich gut für unzuverlässige Netze mit geringer Bandbreite und hoher
Latenzzeit. \ac{mqtt} bietet eine hohe Zuverlässigkeit und minimiert die
genutzte Bandbreite. Aufgrund dieser Eigenschaften wird \ac{mqtt} z.~B.
bei Sensornetzwerken, Machine to Machine, Telemedizin, Patientenüberwachung und
dem "`Internet der Dinge"' eingesetzt.

\subparagraph{}
Das Protokoll wurde 1999 von IBM für die
Satellitenkommunikation entwickelt. Später wurde es auch in vielen weiteren
industriellen Anwendungen eingesetzt.

\subparagraph{}
\ac{mqtt} ist einfach zu implementieren. Es kann ununterbrochene Sitzungen mit
Geräten betreiben. Seine Struktur besteht aus einem Publisher, einem Subscriber
und einem Message-Broker, der für die Kommunikationssteuerung sorgt.
% TODO erklären was Publisher, Subscriber und Message-Broker sind
\cite{dennisseidel2018}
