\chapter{Umsetzung}\label{sec:umsetzung}
\section{Auswahl des Roboters}
In Kapitel \ref{sec:moderne-humanoide-roboter} auf Seite
\pageref{sec:moderne-humanoide-roboter} werden verschiedene humanoide Roboter
vorgestellt. Nun soll der Roboter ausgewählt werden, der am besten dafür
geeignet ist PowerPoint-Präsentationen vorzustellen.

\subsection{Fähigkeiten}
Vorraussetzungen für das Präsentieren von PowerPoint-Folien sind eine
Sprachausgabe und eine Möglichkeit die einzelnen Folien anzuzeigen. Beides kann
zwar durch die Anbindung eines externen Lautsprechers bzw. Bildschirms ersetzt
werden, da der eingesetzte Roboter in der Fiducia \& GAD jedoch nur ähnliche
Aufgaben übernimmt, die die gleichen Vorraussetzungen haben, würde ein Roboter
ohne Lautsprecher oder ein Display die Umsetzung der Anwendung nur unnötig
komplizierter machen.

\subparagraph{}
Damit sind die Roboter Atlas und Sophia nicht die optimale Lösung. Bei Atlas
wird der Fokus darauf gelegt, dass er sich möglichst stabil bewegen kann. Auf
andere Fähigkeiten, die ihn menschlich machen wurde keine Rücksicht genommen. So
besitzt er weder Lautsprecher noch Displays. Da der Roboter sich nur auf
bekanntem und geradem Gelände bewegen wird (die Präsentationen werden auf
Messen oder in Banken mit festem Boden, nicht z.~B. im Wald vorgetragen) bringt
die Fähigkeit, sich sicher auf den Beinen zu halten, keinen entscheidenten
Vorteil, der den Einsatz von Atlas gerechtfertigen würde.

\subparagraph{}
Sophia soll einem Menschen möglichst ähnlich sein. Damit besitzt der Roboter
Lautsprecher zur Sprachausgabe, jedoch selbstverständlich kein eingebautes
Display.

\subparagraph{}
Die anderen Roboter (Hub-Robot, Paul und Pepper) sind jeweils mit Lautsprechern
und Displays ausgestattet. Auch falls weitere Fähigkeiten für die Erweiterung
notwendig werden, sind diese drei Roboter gleich gut ausgestattet. Zum Beispiel
haben sie die Möglichkeit sich zu bewegen. Außerdem verfügen sie über eine
Verbindung zum Internet um Dialogsysteme, wie z.~B. Microsoft Azure, IBM Watson,
Google Home oder Alexa von Amazon verwenden zu können.

\subsection{Wirkung auf Kunden}
Da das Ziel ist, mit dem Roboter Kunden anzulocken und Kunden mit dem Roboter
interagieren sollen, ist die Wirkung, die der Roboter auf einen Menschen hat bei
der Auswahl des Roboters zu beachten. Die Menschen verbinden mit verschiedenen
Robotern verschiedene Emotionen, auf diese muss Rücksicht genommen werden.

\subparagraph{}
Boston Dynamics bezeichnet seine Roboter selbst als "`Albtraum-Auslöser"'
("`nightmare-inducing"' \cite{Guardian2018}). Zwar löst Atlas eine große
Faszination aus, wenn er sich bewegt wie ein Mensch, sich nicht umwerfen lässt
und sogar Saltos machen kann. Allerdings löst er auch skepsis aus. Atlas ist
robust gebaut und wirkt eher aggresiv als einladend. Nicht zuletzt weil er auch
für militärische Zwecke eingesetzt werden soll ist Atlas nicht geeignet um
Kunden auf Messen oder in Banken neue Produkte vorzustellen.

\subparagraph{}

\subsection{Fazit}

\section{Struktur}\label{sec:struktur}
Der größte Teil der Anwendung läuf auf einem Flask-Server. Auf diesem Server
werden PowerPoint-Dateien, die vom Benutzer hochgeladen werden können, zur
Präsentation durch Pepper umgewandelt. Dazu werden die einzelnen Folien zu
Bildern zur Darstellung als HTML-Datei konvertiert. Die Notizen der Folien
werden als Text extrahiert. Diesen Text wird Pepper während der Präsentation
passend zur jeweiligen Folie vorlesen. Die Folien werden auf dem Tablet auf
Peppers Brust angezeigt. Welche Präsentation vorgestellt werden soll, kann auf
dem Tablet ausgewählt werden.

\begin{lstlisting}[float, language=Python, frame=single, framexleftmargin=15pt,
style=algoBericht, label={lst:vorlage}, captionpos=b, caption={Vorlage für das
Einfügen von Code-Beispielen}]
print "Hello, world!"
\end{lstlisting}

\section{Server}\label{sec:server}
