\chapter{Einleitung}\label{sec:einleitung}
\section{Die Fiducia \& GAD IT AG}\label{sec:fiducia-gad}
Die Fiducia \& GAD IT AG ist der Dienstleister für Informationstechnologie der
genossenschaftlichen FinanzGruppe. Das Unternehmen beschäftigt aktuell rund
6.400 Mitarbeiter an den Verwaltungssitzen in Karlsruhe und Münster und den
Geschäftsstellen in München, Frankfurt und Berlin. Die Fiducia \& GAD
erwirtschaftet einen jährlichen Konzernumsatz von rund 1,4 Milliarden Euro.

\subparagraph{}
Kunden der Fiducia \& GAD sind alle 1.000 Volksbanken und Raiffeisenbanken in
Deutschland und die Unternehmen der genossenschaftlichen FinanzGruppe. Außerdem
gehören zahlreiche Privatbanken und Unternehmen anderer Branchen, wie z.B. der
ADAC, zum Kundenkreis der Fiducia \& GAD.

\subparagraph{}
Neben dem Betrieb der beiden Bankverfahren "`agree21"' und "`bank21"' in ihren
vier Hochsicherheitsrechenzentren, betreut die Fiducia \& GAD 173.000
Bankarbeitsplätze und verwaltet knapp 83 Millionen Kundenkonten. Außerdem stellt
die Fiducia \& GAD mit 36.000 eigenen Selbstbedienungsgeräten bundesweit eine
reibungslose Bargeldversorgung sicher. \cite{FiduciaGAD2018}

\section{Motivation}\label{sec:motivation}
Die Fiducia \& GAD erforscht Möglichkeiten humanoide Roboter produktiv
eizusetzen. Dazu sollen sie in eigene oder für ihre Kunden bereitgestellte
Geschäftsprozesse eingebunden werden. Zu diesen Zwecken wird der Roboter Pepper
der Firma SoftBank Robotics eingesetzt.

\subparagraph{}
Humanoide Roboter lösen bei vielen Menschen eine große Faszination aus. Obwohl
sie in Japan bereits weit verbreitet sind und mehrere Firmen immer
fortgeschrittenere entwickeln, trifft man im realen Leben nur sehr selten auf
humanoide Roboter. Die meisten Menschen kennen diese Art Roboter nur aus
Science-Fiction Filmen. Diese verbreiten ein faszinierendes, wenn auch teilweise
beängstigendes, Bild von Robotern, die den Menschen im täglichen Leben
unterstützen und dabei in Bewegung, Sprache und Aussehen einem Menschen ähneln.
Doch humanoide Roboter sind nicht mehr nur Science-Fiction. Sie haben das
Potenzial tägliche Begleiter der Menschen zu werden, wie zuletzt der Computer,
oder als noch aktuellere technische Entwicklung, das Smartphone.

\section{Zielsetzung}\label{sec:zielsetzung}
Momentan werden hierbei zunächst komplexere Anwendungsszenarien betrachtet,
beispielsweise eine geführte Kontoeröffnung. Diese werden im Rahmen von
Veranstaltungen auch immer wieder bei Kunden präsentiert. Hierbei ist meist die
Einbindung eigener, zur Veranstaltung passender Inhalte erwünscht. Um dieser
Anforderung und der Usergruppe zu begegnen, soll die Möglichkeit der
Präsentation von PowerPoint-Folien durch den Roboter geschaffen werden.

\subparagraph{}
Ziel ist es eine Online-Plattform zu erstellen, die PowerPoint-Dateien so
umwandelt, dass die Folien auf einem Tablet, welches auf dem Bauch des
Roboters angebracht ist, angezeigt werden können.
Zusätzlich soll Pepper die Notizen der einzelnen Folien jeweils zur
entsprechenden Folie vorlesen. Dazu wird eine Anwendung entwickelt, welche die
Folien zu Bildern und Text umwandelt. Außerdem wird eine App entwickelt, die auf
dem Roboter installiert wird. So kann der Roboter die umgewandelte und
bereitgestellte Präsentation vortragen.
