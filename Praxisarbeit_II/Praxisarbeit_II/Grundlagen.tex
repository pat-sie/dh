\chapter{Grundlagen}\label{sec:grundlagen}
\section{Python}\label{sec:python}
"`Python ist eine portable, interpretative, objektorientiere
Programmiersprache"' \cite{Weigend2017}. Entwickelt wurde sie 1989 von Guido van
Rossum. Heute wird die Entwicklung von der nichtkommerziellen Organisation
"`Python Software Foundation"' \footnote{\url{https://www.python.org/psf/}}
koordiniert.

\subparagraph{}
Das Python-Skript wird von einem Interpreter ausgef�hrt. Python-Skripte k�nnen
auf verschiedenen Systemplattformen (Unix, Windows, Mac OS) laufen, deshalb
bezeichnet man Python als portable Sprache. Die aktuelle Version ist 3.6. Da auf
dem Roboter allerdings Python 2.7 installiert ist, wird die Anwendung mit dieser
Version entwickelt. Python ist kostenlos. \cite{Weigend2017}

\section{JavaScript}\label{sec:javascript}
JavaScript ist, wie der Name bereits sagt, eine Skriptsprache. Verwendet wird
sie haupts�chlich im Worl Wide Web. Mit JavaScript lassen sich Objekte, wie zum
Beispiel das Browserfenster, beeinflussen. \cite{Steyer2010}

\section{PowerPoint}\label{sec:powerpoint}


\section{Flask}\label{sec:flask}
Flask ist ein Python Package zum erstellen eines Servers.