\chapter{Einleitung}\label{sec:einleitung}
\section{Die Fiducia \& GAD IT AG}\label{sec:fiducia-gad}
Die Fiducia \& GAD IT AG ist der Dienstleister f�r Informationstechnologie der
genossenschaftlichen FinanzGruppe. Das Unternehmen besch�ftigt aktuell rund
6.400 Mitarbeiter an den Verwaltungssitzen in Karlsruhe und M�nster und den
Gesch�ftsstellen in M�nchen, Frankfurt und Berlin. Die Fiducia \& GAD
erwirtschaftet einen j�hrlichen Konzernumsatz von rund 1,4 Milliarden Euro.

\subparagraph{}
Kunden der Fiducia \& GAD sind alle 1.000 Volksbanken und Raiffeisenbanken in
Deutschland und die Unternehmen der genossenschaftlichen FinanzGruppe. Au�erdem
geh�ren zahlreiche Privatbanken und Unternehmen anderer Branchen, wie z.B. der
ADAC, zum Kundenkreis der Fiducia \& GAD.

\subparagraph{}
Neben dem Betrieb der beiden Bankverfahren "`agree21"' und "`bank21"' in ihren
vier Hochsicherheitsrechenzentren, betreut die Fiducia \& GAD 173.000
Bankarbeitspl�tze und verwaltet knapp 83 Millionen Kundenkonten. Au�erdem stellt
die Fiducia \& GAD mit 36.000 eigenen Selbstbedienungsger�ten bundesweit eine
reibungslose Bargeldversorgung sicher. \cite{FiduciaGAD2018}

\section{Motivation}\label{sec:motivation}
Humanoide Roboter l�sen bei vielen Menschen eine gro�e Faszination aus. Obwohl
sie in Japan bereits weit verbreitet sind und mehrere Firmen immer
fortgeschrittenere entwickeln, trifft man im realen Leben nur sehr selten auf
humanoide Roboter. Die meisten Menschen kennen diese Art Roboter nur aus
Science-Fiction Filmen. Diese verbreiten ein faszinierendes, wenn auch teilweise
be�ngstigendes, Bild von Robotern, die den Menschen im t�glichen Leben
unterst�tzen und dabei in Bewegung, Sprache und Aussehen einem Menschen �hneln.
Doch humanoide Roboter sind nicht mehr nur Science-Fiction. Sie haben das
Potenzial t�gliche Begleiter der Menschen zu werden, wie zuletzt der PC, oder
als noch aktuellere technische Entwicklung, das Smartphone.

\subparagraph{}
Die Fiducia \& GAD erforscht M�glichkeiten humanoide Roboter produktiv
eizusetzen. Dazu sollen sie in eigene oder f�r ihre Kunden bereitgestellte
Gesch�ftsprozesse eingebunden werden. Zu diesen Zwecken wird der Roboter Pepper
von Softbank Robotics eingesetzt.

\section{Zielsetzung}\label{sec:zielsetzung}
Momentan werden hierbei zun�chst komplexere Anwendungsszenarien betrachtet,
beispielsweise eine gef�hrte Kontoer�ffnung. Diese werden im Rahmen von
Veranstaltungen immer auch wieder bei Kunden pr�sentiert. Hierbei ist meist die
Einbindung eigener, zur Veranstaltung passender Inhalte erw�nscht. Um dieser
Anforderung und der Usergruppe zu begegnen soll die M�glichkeit der
Pr�sentation von PowerPoint-Folien durch den Roboter geschaffen werden.

\subparagraph{}
Ziel ist es eine Online-Plattform zu erstellen, die PowerPoint-Dateien so
umwandelt, dass die Folien auf dem Tablet des Roboters angezeigt werden k�nnen.
Zus�tzlich soll Pepper die Notizen der einzelnen Folien jeweils zur
entsprechenden Folie vorlesen. Dazu wird ein Programm geschrieben, welches die
Folien zu Bildern und Text umwandelt. Au�erdem wird eine App entwickelt, die auf
dem Roboter installiert wird und die den Roboter die, von der Online-Plattform
umgewandelte und bereitgestellte Pr�sentation vortragen l�sst.
