%%%%%%%%%%%%%%%%%%%%%%%%%%%%%%%%%%%%%%%%%%%%%%%%%%%%%%%%%%%%%%%%%%%%%%%%%%%%%%%
%% Descr:       Vorlage f�r Berichte der DHBW-Karlsruhe, Erklärung
%% Author:      Prof. Dr. J�rgen Vollmer, vollmer@dhbw-karlsruhe.de
%% $Id: erklaerung.tex,v 1.6 2016/03/16 12:51:09 vollmer Exp $
%% -*- coding: utf-8 -*-
%%%%%%%%%%%%%%%%%%%%%%%%%%%%%%%%%%%%%%%%%%%%%%%%%%%%%%%%%%%%%%%%%%%%%%%%%%%%%%%

% In Bachelorarbeiten muss eine schriftliche Erkl�rung abgegeben werden.
% Hierin bestätigen die Studierenden, dass die Bachelorarbeit, etc.
% selbständig verfasst und s�mtliche Quellen und Hilfsmittel angegeben sind. Diese Erklärung
% bildet das zweite Blatt der Arbeit. Der Text dieser Erkl�rung muss auf einer separaten Seite
% wie unten angegeben lauten.

\newpage
\thispagestyle{empty}
\begin{framed}
\begin{center}
\Large\bfseries Erkl�rung
\end{center}
\medskip
\noindent
Ich versichere hiermit, dass ich meine \Was\ mit
dem Thema: \enquote{\Titel} selbstst�ndig verfasst und keine anderen als die angegebenen Quellen und
Hilfsmittel benutzt habe. Ich versichere zudem, dass die eingereichte elektronische Fassung mit der
gedruckten Fassung �bereinstimmt.

\vspace{3cm}
\noindent
\underline{\hspace{4cm}}\hfill\underline{\hspace{6cm}}\\
Ort~~~~~Datum\hfill Unterschrift\hspace{4cm}
\end{framed}

\vfill
\begin{framed}
\begin{center}
\Large\bfseries Sperrvermerk
\end{center}
\medskip
\noindent
Der Inhalt dieser Arbeit darf weder als Ganzes noch in Ausz�gen Personen
au�erhalb des Pr�fungsprozesses und des Evaluationsverfahrens zug�nglich gemacht
werden, sofern keine anders lautende Genehmigung der Ausbildungsst�tte vorliegt.
\end{framed}

%%%%%%%%%%%%%%%%%%%%%%%%%%%%%%%%%%%%%%%%%%%%%%%%%%%%%%%%%%%%%%%%%%%%%%%%%%%%%%%
\endinput
%%%%%%%%%%%%%%%%%%%%%%%%%%%%%%%%%%%%%%%%%%%%%%%%%%%%%%%%%%%%%%%%%%%%%%%%%%%%%%%
