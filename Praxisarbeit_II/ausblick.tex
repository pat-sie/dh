\chapter{Ausblick}\label{sec:ausblick}
Aktuell laufen die Flask-Anwendung und der \ac{mqtt} Message Broker lokal auf
einem Computer. Im produktiven Einsatz soll es für einen Kunden möglich sein über das
Internet eine PowerPoint-Datei hochzuladen und diese einfach durch Pepper
vorstellen zu lassen. Dazu muss die Anwendung auf einem Server laufen.

\subparagraph{}
Potenziell besteht die Möglichkeit mehrere Roboter einzusetzen. Durch die
Verwendung von \ac{mqtt} erhält der Roboter eine Nachricht vom Server, welche
Präsentation vorgestellt werden soll. Danach läuft die Anwendung unabhängig vom
Server. Da die Fiducia \& GAD derzeit nur im Besitzt eines Exemplars von Pepper
ist, ist eine Unterscheidung zwischen verschiedenen Robotern nicht nötig. In der
Zukunft kann die Anwendung aber ohne großen Aufwand so erweitert werden, dass
mehrere Roboter gleichzeitig verschiedene (oder auch die gleiche) Präsentationen
vorstellen können. Falls sich eine oder mehrere Banken für den dauerhaften
Einsatz von Pepper in einer Filliale entscheiden, ist diese Funktion nötig.

\subparagraph{}
Um nicht zwei verschiedene Anwendung auf Pepper installieren zu müssen, kann das
Präsentieren von PowerPoint-Folien als zusätzliches Feature in die aktuelle
Anwendung, den "`Pepper Bot"' implementiert werden. Damit müssen für
verschiedene Anwendungsfälle nicht verschieden Programme gestartet werden. Die
Anwendungen auf Pepper würden vereinheitlicht werden. Außerdem werden im
"`Pepper Bot"' bereits IBM Watson, Microsoft Azure und MQTT verwendet. So ist es
möglich diese Funktionen auch während der Präsentation zu verwenden ohne erneut
den Aufwand, den dies benötigt, aufbringen zu müssen.
